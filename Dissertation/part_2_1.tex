\chapter{Математические модели поглощения гелия микросферами и сорбентом на их основе}
\section{Математические модели поглощения гелия микросферами и сорбентом на их основе в статических условиях}
\subsection{Модель типа <<растворение-диффузия>> для изучения проникновения гелия внутрь микросфер}

Избирательная проницаемость гелия через стенку микросферы происходит в следствие малости кинетического диаметра молекулы гелия и соответствует разделу газоразделения с помощью непористых мембран \cite{Mulder}. Простейшим способом описания этого явления является использования закона Фика для диффузии гелия внутри стенки и закона Генри для определения количества растворённого вещества на границе мембраны. Таким образом, ставится следующая математическая задача для диффузии гелия через стенку микросферы в сферическом приближении:
\begin{equation}
\label{eq:FikHenry_hellium_permeation_model}
\pd{c}{t}=\frac{D}{r^2}\pd{}{r}\left( r^2 \pd{c}{r} \right),
\end{equation}
где $c(t,r)$ -- концентрация гелия в стенке микросферы в зависимости от времени и координаты ($t > 0$, $a \leq r \leq b$); $D$ -- коэффициент диффузии гелия в материале стенки микросферы; $a$, $b$ -- радиусы полости и внешний радиус микросферы соответственно. 

Рассматривая задачу диффузии в квазистационарном приближении (когда профиль концентрации устанавливается быстро при заданных на границах концентрациях газа), пренебрегают производной по времени 
\begin{equation}
\label{eq:FikHenry_hellium_permeation_model_quasistat}
c = c(r)
\end{equation}
и к уравнению (\ref{eq:FikHenry_hellium_permeation_model}) добавляют следующие граничные условия:
\begin{equation}
\label{eq:FikHenry_hellium_permeation_model_conditions}
c(t, a)  = k_S p_2,\quad
c(t, b)  = k_S p_1,\quad
\end{equation}
где $p_1$, $p_2$ -- давления гелия снаружи и внутри микросферы соответственно; $k_S$~--~коэффициент растворимости гелия в материале стенки микросферы в соответствии с законом Генри.

Решением (\ref{eq:FikHenry_hellium_permeation_model}) с учётом (\ref{eq:FikHenry_hellium_permeation_model_quasistat}) и (\ref{eq:FikHenry_hellium_permeation_model_conditions}) является выражение, описывающее профиль концентрации:
\begin{equation}
\label{eq:eq:FikHenry_hellium_permeation_model_solution}
c(r) = -\frac{k_S(p_1-p_2)ab}{(b-a)r}+\frac{k_S(bp_1-ap_2)}{b-a}.
\end{equation}

Массовый поток гелия через любое сечение $r=r_0$ ($a\leq r_0 \leq b$) с площадью $S_0=4\pi r_0^2$ равен
\begin{equation}
\label{eq:FikHenry_hellium_permeation_model_massflow}
q = -D \left. \pd{c}{r} \right|_{r=r_0}S_0=-\frac{C_m S \beta}{d}(p_1-p_2),
\end{equation}
где $C_m = D k_s$ -- коэффициент проницаемости материала стенки; $S=4 \pi b^2$~--~площадь поверхности микросферы; $\beta$ = $a/b$; $d=b-a$ -- толщина стенки микросферы.

Массовый поток для сферической мембраны (\ref{eq:FikHenry_hellium_permeation_model_massflow}) отличается от потока для плоской мембраны \cite{Hvang, Ditnerskiy, Mulder} множителем $\beta$. В случае, когда стенки микросфер очень тонкие $a \approx b$ им можно пренебречь.

\subsection{Математическая модель поглощения гелия в предположении одинаковости физических и геометрических свойств микросфер}
\subsection{Математическая модель поглощения гелия в предположении дисперсионного распределения микросфер по геометрическим и физическим параметрам}
\subsection{Моделирование поглощения гелия композитным сорбентом на основе микросфер}

