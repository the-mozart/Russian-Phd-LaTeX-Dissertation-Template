\chapter{Обзор} \label{chapt_1}
\section{Обзор по истории гелия и методам его выделения} \label{section_1_1}

Гелий -- второй элемент в таблице Менделева, самый распространённый  изотоп $^4$He которого имеет среднюю атомную массу  $4,0026$~а.е.м., был обнаружен в 1968 году независимо учёными Жансеном и Локьером в спектральными методами при исследовании солнечной короны. В 1871 году Кельвин предложил назвать обнаруженное вещество <<гелий>>.
В 1895 году Рамзай при исследовании газа, выделенного из минерала клевеита, обнаружил присутствие гелия на Земле. Было установлено, что неопознанная линия спектра $D_3$, отвечавшая новому элементу, имела длину волны $5874,9$~\AA \cite{Fastovskii}.

Гелий (в основном $^4$He) в земной атмосфере (земной гелий) -- продукт $\alpha$-распада тяжелых радиоктивных элементов  (U, Th, Ac). Скорость образования гелия мала --- за один год $1$~т урана, связанного минералами, выделяет около $0,12$~см$^3$ гелия. Далее он остаётся в земной коре (в природном газе) либо рассеивается из атмосферы в космос. Содержание другого стабильного изотопа $^3$He крайне мало как в воздухе, так и в природном газе. Соотношения содержания  $^3$He/$^4$He составляет $1,1\cdot 10^{-6}$ для воздуха и $1,4\cdot 10^{-7}$ для природного газа \cite{Fastovskii}. Низкая скорость образования гелия объясняет низкое содержание гелия в природном газе и атмосфере.

Особые свойства \cite{Yakuceni_Helium}

Области применения и использование гелия  \cite{Yakuceni_Helium, Yakuceni_USA}

Сырьевая база гелия \cite{Yakuceni_Material_Base}


\subsection{Мембранный метод для получения гелий-концентрата} \label{section_1_1_1}

\subsection{Адсорбционные методы очистки гелия от примесей} \label{section_1_1_2}

\subsection{Гибридные методы получения чистого гелия} \label{section_1_1_3}