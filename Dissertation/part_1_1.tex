\chapter{Обзор} \label{chapt_1}
\section{Обзор по истории открытия гелия и методам его выделения} \label{section_1_1}

Гелий -- второй элемент в таблице Менделева, самый распространённый  изотоп $^4$He которого имеет среднюю атомную массу  $4,0026$~а.е.м., был обнаружен в 1968 году независимо учёными Жансеном и Локьером в спектральными методами при исследовании солнечной короны. В 1871 году Кельвин предложил назвать обнаруженное вещество <<гелий>>.
В 1895 году Рамзай при исследовании газа, выделенного из минерала клевеита, обнаружил присутствие гелия на Земле. Было установлено, что неопознанная линия спектра $D_3$, отвечавшая новому элементу, имела длину волны $5874,9$~\AA \cite{Fastovskii}.

Гелий относится к группе <<инертных>> газов. Особые свойства гелия: легкость (легче только водород H$_2$); абсолютная инертность; низкая адсорбционная способность; низкая растворимость в пластовых водах; высокая диффузионная способность и проницаемость, вызванная малым диаметром атомов ($2,7$~\AA). Стоит также отметить сверхтекучесть гелия (He-II) ниже температуру кипения $2,186$~К \cite{Yakuceni_Helium}.

Гелий имеет огромную ценность из-за своих уникальных свойств. 
Основные области применения гелия \cite{Yakuceni_Helium, Yakuceni_USA}:
\begin{itemize}
\item сверхпроводимость (включая МРТ) -- 29~\%;
\item воздухоплавание  -- 16~\%;
\item сварка и резка металлов -- 12~\%;
\item оптико-волокно -- 7~\%;
\item аналитические цели  -- 6~\%;
\item атомная энергетика  -- 6~\%;
\item детектирование микротечей  -- 6~\%;
\item полупроводники  -- 5~\%;
\item ракетная техника  -- 4~\%;
\item выплавка металлов  -- 3~\%;
\item дыхательные смеси  -- 2~\%;
\item другие  -- 4~\%.	
\end{itemize}

Вследствии своей лёгкости и <<текучести>> после использования он улетучивается в атмосферу, и не подлежит утилизации с возможностью последующего использования. Поэтому необходимы эффективные способы выделения и хранения гелия из имеющихся ресурсов этого газа.


Гелий (в основном $^4$He) в земной атмосфере (земной гелий) -- продукт $\alpha$-распада тяжелых радиоктивных элементов  (U, Th, Ac). Скорость образования гелия мала --- за один год $1$~т урана, связанного минералами, выделяет около $0,12$~см$^3$ гелия. Далее он остаётся в земной коре (в природном газе) либо рассеивается из атмосферы в космос. Содержание другого стабильного изотопа $^3$He крайне мало как в воздухе, так и в природном газе. Соотношения содержания  $^3$He/$^4$He составляет $1,1\cdot 10^{-6}$ для воздуха и $1,4\cdot 10^{-7}$ для природного газа \cite{Fastovskii}. Низкая скорость образования гелия объясняет низкое содержание гелия в природном газе и атмосфере.

Как правило на Земле гелий добывают из природного газа, т.к. содержание гелия в атмосфере ничтожно мало. Месторождения природного газа по содержанию гелия делятся на три основых типа:
\begin{itemize}
	\item бедные (меньше $0,01$~\%);
	\item рядовые (от $0,01$~\% до $0,5$~\%);
	\item богатые (выше $0,5$~\%).	
\end{itemize}



\todo{Сырьевая база гелия \cite{Yakuceni_Material_Base}}

\subsection{Криогенный метод извлечения гелия из природного газа} \label{section_1_1_1}

ОАО «НПО «Гелиймаш» было разработано и выпущено подавляющее большинство криогенных гелиевых установок, работающих в России и странах ближнего зарубежья. Всего, начиная с первого гелиевого ожижителя, выпущенного в 60-е годы ХХ века, предприятием было поставлено около ста криогенных гелиевых установок различной производительности для научных центров, промышленных предприятий, исследовательских комплексов, больниц, нужд предприятий оборонного комплекса. На базе турбодетандеров собственной разработки с использованием результатов экспериментальных и теоретических работ были созданы криогенные гелиевые установки КГУ-500/4,5-140, КГУ-1600/3,8, КГУ-600/20, ориентированные, как на ожижительных, так и на рефрижераторный режимы на различных температурных уровнях от 3,8~К до 20~К. «НПО «Гелиймаш» создал гелиевые ожижители производительностью 700 литров/час по жидкому гелию для крупнейшего гелиевого центра Европы – Оренбургского гелиевого завода. Одна из последних криогенных гелиевых установок была поставлена  Корпорации «ТВЭЛ» для тестирования сверхпроводящих элементов, предназначенных для европейского международного проекта термоядерного реактора ИТЭР в Кадараше. Создан ожижитель производительностью 200~л/ч, ведутся работы по созданию гелиевого ожижителя производительностью 1100~л/ч для СКО международного проекта NICA в ОИЯИ (Дубна). В крупных ожижителях гелия применены одноступенчатые жидкостно-паровые турбодетандеры вместо дроссельных вентилей, что позволило значительно увеличить производительность установок. Аналогичными турбодетандерами оснащены три ожижителя гелия, которые более шести лет успешно работают в г. Оренбурге, а также крупный ожижитель гелия в Российском научном центре «Курчатовский институт» (Москва). ОАО «ГАЗПРОМ» - ОАО «НПО «Гелиймаш» в период 1980 -- 1995~гг реализовали крупномасштабные российские технологии (до 10~млн.нм$^3$/год) по выделению, очистке и ожижению гелия.Все мощности по ожижению гелия на ОГЗ реализованы с помощью технологий и оборудования «НПО <<Гелиймаш>> \cite{GeliyMash_OG-1000}.


\cite{Stepanov_Avtoreferat}

\subsection{Мембранный метод для получения гелий-концентрата} \label{section_1_1_1}

\subsection{Адсорбционные методы очистки гелия от примесей} \label{section_1_1_2}

\subsection{Гибридные методы получения чистого гелия} \label{section_1_1_3}