\chapter{Обзор} \label{chapt_1}
\section{Обзор по истории гелия и методам его выделения} \label{section_1_1}

Гелий -- второй элемент в таблице Менделева, самый распространённый  изотоп $^4He$ которого имеет среднюю атомную массу  $4,0026$~а.е.м., был обнаружен в 1968 году независимо учёными Жансеном и Локьером в спектральными методами при исследовании солнечной короны. В 1871 году Кельвин предложил назвать обнаруженное вещество <<гелий>>.
В 1895 году Рамзай при исследовании газа, выделенного из минерала клевеита, обнаружил присутствие гелия на Земле. Было установлено, что неопознанная линия спектра $D_3$, отвечавшая новому элементу, имела длину волны $5874,9$~\AA \cite{Fastovskii}.

Особые свойства \cite{Yakuceni_Helium}

Области применения и использование гелия  \cite{Yakuceni_Helium, Yakuceni_USA}

Сырьевая база гелия \cite{Yakuceni_Material_Base}


\subsection{Мембранный метод для получения гелий-концентрата} \label{section_1_1_1}

\subsection{Адсорбционные методы очистки гелия от примесей} \label{section_1_1_2}

\subsection{Гибридные методы получения чистого гелия} \label{section_1_1_3}